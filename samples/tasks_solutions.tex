\documentclass[a4paper,12pt]{article}
\usepackage[utf8]{inputenc}
\usepackage[T2A]{fontenc}
\usepackage[english,russian]{babel}
\usepackage{amsmath,amssymb}
\usepackage{geometry}
\geometry{left=2cm,right=2cm,top=2cm,bottom=2cm}
\begin{document}

\section*{Сгенерированные задачи ЛП и их решения}

\subsection*{Задача №1}
\textbf{Функция цели: }
maximize $ -5x_{1} +0x_{2} +5x_{3} $\\

\textbf{Ограничения:}

\[ \begin{aligned}
4x_{1} +1x_{2} -5x_{3} &= 4 \\ 
-3x_{1} +5x_{2} -3x_{3} &= 4 \\ 
2x_{1} -3x_{2} -2x_{3} &= 3 \\ 
x_i &\ge 0, \quad i=1,\dots,3 \\ 
\end{aligned}\]

\textbf{Задача не имеет решения или возникла ошибка.}

\textbf{Шаги симплекс-метода (упрощённо):}

\begin{verbatim}
=== Шаги симплекс-метода (демонстрация) ===
1) Составляем симплекс-таблицу на основе ограничений.
2) Ищем ведущий столбец и строку.
3) Пересчитываем таблицу.
4) Проверяем условие оптимальности.
5) Продолжаем, пока не достигнем оптимума или нет решения.

\end{verbatim}



\subsection*{Задача №2}
\textbf{Функция цели: }
maximize $ -5x_{1} +1x_{2} +0x_{3} +4x_{4} $\\

\textbf{Ограничения:}

\[ \begin{aligned}
5x_{1} -2x_{2} +5x_{3} +4x_{4} &= 4 \\ 
3x_{1} +1x_{2} +2x_{3} +1x_{4} &\le 4 \\ 
x_i &\ge 0, \quad i=1,\dots,4 \\ 
\end{aligned}\]

\textbf{Оптимальное значение: }$10.00$

\textbf{Решение: }$x_{1}=0.00, x_{2}=2.00, x_{3}=0.00, x_{4}=2.00$

\textbf{Шаги симплекс-метода (упрощённо):}

\begin{verbatim}
=== Шаги симплекс-метода (демонстрация) ===
1) Составляем симплекс-таблицу на основе ограничений.
2) Ищем ведущий столбец и строку.
3) Пересчитываем таблицу.
4) Проверяем условие оптимальности.
5) Продолжаем, пока не достигнем оптимума или нет решения.

\end{verbatim}



\subsection*{Задача №3}
\textbf{Функция цели: }
maximize $ 1x_{1} +5x_{2} +3x_{3} -1x_{4} $\\

\textbf{Ограничения:}

\[ \begin{aligned}
-1x_{1} +4x_{2} -3x_{3} -4x_{4} &\ge 2 \\ 
4x_{1} +3x_{2} -1x_{3} +1x_{4} &= 1 \\ 
x_i &\ge 0, \quad i=1,\dots,4 \\ 
\end{aligned}\]

\textbf{Задача не имеет решения или возникла ошибка.}

\textbf{Шаги симплекс-метода (упрощённо):}

\begin{verbatim}
=== Шаги симплекс-метода (демонстрация) ===
1) Составляем симплекс-таблицу на основе ограничений.
2) Ищем ведущий столбец и строку.
3) Пересчитываем таблицу.
4) Проверяем условие оптимальности.
5) Продолжаем, пока не достигнем оптимума или нет решения.

\end{verbatim}



\end{document}